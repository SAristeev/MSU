\documentclass{article}
\usepackage[T2A]{fontenc}
\usepackage[utf8]{inputenc}
\usepackage[english,russian]{babel}
\begin{document}
\begin{center}
\Large\bf Простая задача на работу с BMP-изображениями в Python
\end{center}

Формат вызова программы должен быть следующим:

./prog InputFile1 InputFile2 OutputFile

Программа должна загрузить изображения из графических файлов
InputFile1 и InputFile2, поместить второе изображение в центр первого
изображения

{\large\bf Решение:}

Установим сначала библиотеку Pillow. Подключим также модуль Sys для работы с аргументами командной строки

Изображение для нас это список, индексируемый парой чисел: высотой и шириной. Элементом же этого списка является тройка чисел (RGB). Создадим новый список по размерам равный первому изображению.

Для каждого пикселя нового изображения будем копировать пиксель из нужного изображения(в зависимости от расстояния до центра фото)

Сохраним новое изображение в OutputFile


\end{document}