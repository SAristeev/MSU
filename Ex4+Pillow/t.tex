\documentclass{article}
\usepackage[T2A]{fontenc}
\usepackage[utf8]{inputenc}
\usepackage[english,russian]{babel}
\begin{document}
\begin{center}
\Large\bf Простая задача на работу с BMP-изображениями в Python
\end{center}

Формат вызова программы должен быть следующим\\
./prog N InputFile OutputFile

Программа должна загрузить изображение из графического файла InputFile, создать вокруг введенного изображения черную рамку ширины N (ширина и высота изображения, при этом, увеличиваются на 2N)

{\large\bf Решение:}

Установим сначала необходимые библиотеки: SciPy, NumPy, Imageio. Подключим также модуль Sys для работы с аргументами командной строки

Изображение для нас это трехмерный список: высота, ширина, RGB. Создадим новый список длиной на 2N больше длины исходного избражения. Каждый элемент этого списка - список также на 2N больше. (Высота и ширина соответственно).
Первые и последние N списков заполним черным цветом. То есть каждая ячейка первых N строк будет списком из трех нулей - интенсивность каждого цвета.

Остальные списки сначала заполним N черных ячеек, после скопируем ячейку из начального изображения. Ее координаты меньше на N. Аналогично последние ячейки заполним черным цветом

Сохраним новое изображение в OutputFile


\end{document}